%%%%%%%%%%%%%%%%%%%%%%%%%
% PACKAGES              %
%%%%%%%%%%%%%%%%%%%%%%%%%
\documentclass{report}

\usepackage[utf8x]{inputenc}  % accents
\usepackage{geometry}         % marges
\usepackage[francais]{babel}  % langue
\usepackage{graphicx}         % images
\usepackage{verbatim}         % texte préformaté
\usepackage{fancyhdr}         % fancy
\usepackage{listings}         % source code 




%%%%%%%%%%%%%%%%%%%%%%%%%
% PRÉAMBULE             %
%%%%%%%%%%%%%%%%%%%%%%%%%
\title{Delaunator Architecture and basic Specifications} 
\author{}
% laisser vide pour date de compilation
\date{} 

% FORMAT PAGES         
%\pagestyle{fancy} % nom du rendu (définit les lignes suivantes)
        %\lhead{} % left head
        %\chead{} % center head
        %\rhead{} % right head
        %\lfoot{} % left foot
        %\cfoot{} % center foot
        %\rfoot{} % right foot







%%%%%%%%%%%%%%%%%%%%%%%%%
% BEGIN                 %
%%%%%%%%%%%%%%%%%%%%%%%%%
\begin{document}
        \maketitle




%%%%%%%%%%%%%%%%%%%%%%%%%
% CHAPTER 1             %
%%%%%%%%%%%%%%%%%%%%%%%%%
\chapter{Introduction}
    \paragraph*{}
    Delaunator is a python module designed for one main objective: provide in quick time a reference to neighbors objects in a 2D world.
    There is a lots of way to do that, and the importants one are implemented, notabily access on all objects away from a certain distance, 
    or the N closer objects.

    \paragraph*{}
    Delaunator follow some way for do its job:
    \begin{enumerate}
        \item[\textbf{Efficient:}] implementing solutions with minimal algorithmic complexity and critical parts coded in low-level language.
        \item[\textbf{Pythonic:}] support both of Python 2 and 3, and provide a simple and elegant API as Python module.
        \item[\textbf{Strong:}] robust management of limit cases and \textit{good ideas} of user.
        \item[\textbf{Non-invasive:}] don't need complicated initialisation by user and don't asking questions; its just give answers.
    \end{enumerate}
    \newpage
    \begin{lstlisting}[language=python]
    from random import randint
    from delaunator import Delaunator, TrianguledObject

    # useful class
    class Student(TrianguledObject): 
        def __init__(self, name):
            super().__init__()
            self.name = str(name)
        def __str__(self):
            return self.name

    # bounds
    xmin, xmax, ymin, ymax = 0, 600, 0, 600

    # creat delaunator and data
    dt = Delaunator(xmin, xmax, ymin, ymax)
    michel = Student('michel')
    dt.addTrianguledObject(michel, 342, 123)

    for i in range(100):
        dt.addTrianguledObject(
            randint(xmin, xmax), 
            randint(ymin, ymax), 
            Student('totoro')
        )

    # movement
    dt.moveTrianguledObject(michel, randint(xmin, xmax) / 2, randint(ymin, ymax) / 2)
    
    # print name of all neighbors that are to a distance at most 100
    print("My neighbors are " + ", ".join([str(_) for _ in michel.neighborsAt(100)]))
    
    # frees
    dt.delTrianguledObject(michel)
    dt.delAllObjects()
    \end{lstlisting}



%%%%%%%%%%%%%%%%%%%%%%%%%
% CHAPTER 2             %
%%%%%%%%%%%%%%%%%%%%%%%%%
\chapter{Global Architecture - low level part}
This part is coded in C++, for rapidity and object oriented feature support.

\section{Delaunay triangulation}
    \paragraph*{}
    Delaunay Triangulation, as a mathematical object implemented in quad-edge implementation, is defined mainly in \textbf{Triangulation}, \textbf{Vertex}, \textbf{Edge} and \textbf{Face} classes.
    \begin{itemize}
        \item[\textbf{Triangulation:}] manager of delaunay triangulation and Vertex, Edge and Face classes.
        \item[\textbf{Vertex:}] node of the graph, where data can be found.
        \item[\textbf{Edge:}] link between vertices of graph, main object of quad-edge implementation.
        \item[\textbf{Face:}] triangulary surface, useless for final user but have lots of useful low-level methods.
    \end{itemize}



\section{Technical Overlay}
    \paragraph*{}
    Additionnal objects are used for involve management of limit cases: \textbf{Delaunator} and \textbf{VirtualVertex}.
    Basically, Delaunator manage VirtualVertex like Triangulation manage Vertex. Difference come where a VirtualVertex instance references a unique Vertex instance, and all is as if a VirtualVertex is placed in a Triangulation
    instance. Moreover, each Vertex have a list of VirtualVertex that references it. 
    \paragraph*{}
    These links allow Delaunator to manage a triangulation where cunfunding data is not a problem, without touch to commons graph exploration algorithms.
    Delaunator is important in module API. (see chapter on API)
    \paragraph*{}
    \begin{itemize}
        \item[\textbf{Delaunator:}] object where VirtualVertex are placed. Provides many methods for managing objects globally.
        \item[\textbf{VirtualVertex:}] managed by Delaunator, referenced by an id that allow a quick association to TrianguledObject.
    \end{itemize}



\section{Other C/C++ modules}
    \paragraph*{}
    \begin{itemize}
        \item[\textbf{Geometry:}] C++ namespace where lots of functions useful for answer to geometric problems.
        \item[\textbf{Utils:}] contain lots of useful functions, templates and declarations for logging, terminal managing, operation shorcuts and optimizations.
    \end{itemize}




%%%%%%%%%%%%%%%%%%%%%%%%%
% CHAPTER 3             %
%%%%%%%%%%%%%%%%%%%%%%%%%
\chapter{Global Architecture - high level part}
    \paragraph*{}
    This part is probably coded in Python, for usability. By an unbelievable chance, this code is not yet implemented, so its to user to do something.
    Author of Delaunator module want to let user define its own projects. (and had just enough time for code Delaunator)




%%%%%%%%%%%%%%%%%%%%%%%%%
% CHAPTER 4             %
%%%%%%%%%%%%%%%%%%%%%%%%%
\chapter{Global Architecture - link level part}
Where C++ and Python run together for merge efficiency of C++ and usability of Python.
Its performed with SWIG, python code and additionnal Cpp-like code. 
The main class of this level is \textbf{Trianguled Object}, and important code is written in all interface files (.i extension).

\section{Trianguled Object}
    \paragraph*{}
    Maybe the most important class for user, Trianguled Object is the final and abstract object of API. 
    In normal case, user defined a class than inherit from TrianguledObject, then add it to a Delaunator class.
    After that, the user have just to use Trianguled Objects methods and code his project.


\section{Delaunator extending}
    \paragraph*{}
    Delaunator C++ class is extended in interface files, with definitions of Trianguled Object. 
    These methods allow API to manage efficiently Trianguled Object.


\section{Interface files}
    \paragraph*{}
    Specifications for SWIG, where low level part accessible by Python will be defined.
    TrianguledObject is defined in interface file, as declarations of shortcuts for help python users. (iteration on C++ lists, accessors as property)
    \paragraph*{}
    By a totally non-professionnal choice, all methods of all objects are accessible from Python.
    Maybe its better to hide low level parts that are not directly in official API. It will be the goal of a next update, maybe.
    





%%%%%%%%%%%%%%%%%%%%%%%%%
% CHAPTER 4             %
%%%%%%%%%%%%%%%%%%%%%%%%%
\chapter{API}
\section{Delaunator object}
    \paragraph*{}
    the main object of API is Delaunator class. Each instance of Delaunator class describe an independant mesh.
    After definition, the only usecases of a delaunator instance is adding and deleting TrianguledObject.
    \paragraph*{}




\end{document}
% END
